 \documentclass[10pt]{article} 
\usepackage[utf8]{inputenc}
\usepackage{geometry}
\geometry{a4paper}
\geometry{top=0.5in, bottom=0.5in, left=.7in, right=.7in}
%\usepackage{blindtext}
\usepackage{hyperref}
\usepackage{bookmark}
\usepackage{booktabs} % for much better looking tables
\usepackage{array} % for better arrays (eg matrices) in maths
%\usepackage{paralist} % very flexible & customisable lists (eg. enumerate/itemize, etc.)
%\usepackage{verbatim} % adds environment for commenting out blocks of text & for better verbatim
\usepackage{subfig} % make it possible to include more than one captioned figure/table in a single float
% These packages are all incorporated in the memoir class to one degree or another...
\usepackage[italian]{babel}

%%% HEADERS & FOOTERS
\usepackage{fancyhdr} % This should be set AFTER setting up the page geometry
\pagestyle{fancy} % options: empty , plain , fancy
\renewcommand{\headrulewidth}{0pt} % customise the layout...
\lhead{}\chead{}\rhead{}
\lfoot{}\cfoot{\thepage}\rfoot{}

%%% SECTION TITLE APPEARANCE
\usepackage{sectsty}
\allsectionsfont{\sffamily\mdseries\upshape} % (See the fntguide.pdf for font help)
% (This matches ConTeXt defaults)

%%% ToC (table of contents) APPEARANCE
\usepackage[nottoc,notlof,notlot]{tocbibind} % Put the bibliography in the ToC
\usepackage[titles,subfigure]{tocloft} % Alter the style of the Table of Contents
\renewcommand{\cftsecfont}{\rmfamily\mdseries\upshape}
\renewcommand{\cftsecpagefont}{\rmfamily\mdseries\upshape} % No bold!

\newtheorem{theorem}{Teorema}
\usepackage{amsmath}
\usepackage{amsfonts}
\usepackage[mode=buildnew]{standalone}% requires -shell-escape
\usepackage{tikz}
\usepackage{pgfplots}
\pgfplotsset{width=10cm,compat=1.17}
\usepackage{pgfplotstable}

\usepackage{multirow}
% \usepackage[keeptemps]{pythontex}
\usepackage{float}
\usepackage{listings}
\lstdefinelanguage{JavaScript}{
  keywords={typeof, new, true, false, catch, function, return, null, catch, switch, var, if, in, while, do, else, case, break},
  keywordstyle=\color{blue}\bfseries,
  ndkeywords={class, export, boolean, throw, implements, import, this},
  ndkeywordstyle=\color{darkgray}\bfseries,
  identifierstyle=\color{black},
  sensitive=false,
  comment=[l]{//},
  morecomment=[s]{/*}{*/},
  commentstyle=\color{purple}\ttfamily,
  stringstyle=\color{red}\ttfamily,
  morestring=[b]',
  morestring=[b]"
}

\lstset{
   language=JavaScript,
   backgroundcolor=\color{lightgray},
   extendedchars=true,
   basicstyle=\footnotesize\ttfamily,
   showstringspaces=false,
   showspaces=false,
   numbers=left,
   numberstyle=\footnotesize,
   numbersep=9pt,
   tabsize=2,
   breaklines=true,
   showtabs=false,
   captionpos=b
}

\newfloat{lstfloat}{htbp}{lop}
\floatname{lstfloat}{Codice sorgente}
\def\lstfloatautorefname{Listato} % needed for hyperref/auroref

\usepackage{animate}


%%% The "real" document content comes below...

\title{Alcuni metodi iterativi per la ricerca di radici di funzioni}
\author{Gionata Massi}
\date{} % Activate to display a given date or no date (if empty),
         % otherwise the current date is printed 

\begin{document}
\maketitle

\thispagestyle{empty}%\frenchspacing

\section{Il problema}

Data una funzione $f : \mathbb{R} \to \mathbb{R}$, determinare un valore reale $\alpha$ tale che $f(\alpha) = 0$.

Usualmente consideriamo funzioni continue in $\mathbb{R}$ o almeno in un intevallo $[a, b] \subseteq \mathbb{R}$ chiuso e limitato in cui ricercare una radice.

\section{Esistenza delle radici}

Non tutte le funzioni ammettono radici, ad esempio $x \mapsto k$ e $x \mapsto (x + k)^2$, dove $k \neq 0$ (es: fig. \ref{fig:no_zeri}).

\begin{figure}[ht]
    \centering
    \includestandalone{img/no_zeri}
    \caption{Funzioni che non intersecano l'asse $y = 0$}
    \label{fig:no_zeri}
\end{figure}
    
Altre funzioni hanno radici nei punti di massimo o di minimo locale (es: fig. \ref{fig:zero_estremante}).

\begin{figure}[ht]
    \centering
    \includestandalone{img/zero_estremanti}
    \caption{Funzioni che intersecano l'asse $y = 0$ in un estremante}
    \label{fig:zero_estremante}
    \end{figure}
    

Per essere sicuri che una funzione ammetta almeno una radice richiediamo che la funzione assuma valori positivi e negativi in un certo intervallo e che sia continua.

\begin{theorem}[Bolzano]
Se $f (x)$ è una funzione continua sull'intervallo limitato e chiuso $[a, b]$ e $f (a) \cdot f (b) < 0$, allora esiste almeno una radice di $f (x)$ nell'intervallo $[a, b]$.
\end{theorem}

Se le ipotesi del teorema sono vere può esistere una sola radice oppure ce ne possono essere in numero finito o anche infinite (fig. \ref{fig:ipotesibolzano}).

\begin{figure}[ht]
    \centering
    \includestandalone{img/segni_discordi}
    \caption{Funzioni che assumono valori opposti agli estremi -1, 1}
    \label{fig:ipotesibolzano}
\end{figure}

Un metodo di ricerca delle radici, se convergente, restituirà una sola delle radici.
Si intuisce che maggiore è la pendenza della funzione in un intorno della radice, più è facile discriminare la radice. Se invece la pendenza è nulla o quasi, allora il problema si dice mal condizionato.

\section{Metodi diretti e metodi iterativi}

I \textbf{metodi diretti} sono algoritmi che, in assenza di errori di arrotondamento, forniscono la soluzione in un numero finito di operazioni.

I \textbf{metodi iterativi} sono algoritmi nei quali la soluzione è ottenuta come limite di una successione di soluzioni. Nella risposta fornita da un metodo iterativo è, quindi, presente usualmente un errore di troncamento.

% Si veda Comincioli2004.

\section{Metodo di bisezione}

Sia $f (x)$ una funzione continua sull'intervallo limitato e chiuso $[a, b]$ con $f (a) \cdot f (b) < 0$. L'algoritmo genera una successione di intervalli
$[a_k , b_k]$ con $f (a_k ) \cdot f (b_k ) < 0$ e con $[a_k , b_k ] \subset [a_{k-1}$ , $b_{k-1} ]$ e $|b_k - a_k | = \frac{1}{2}|b_{k-1} - a_{k-1} |$. 

Date due tolleranze $\epsilon_1$ , $\epsilon_2$ , l’algoritmo si arresta o quando $|b_k - a_k | \leq \epsilon_1$ o quando $|f (\frac {a_k +b_k}{2} )| \leq
\epsilon_2$ o infine quando $k > \text{nmax}$, ove nmax è un numero massimo di iterazioni fissato.

Per alleggerire la notazione usiamo $s_{a_k}$ per indicare $\mathrm{segno} \left(f\left(a_k\right)\right)$,
$s_{b_k}$ per $\mathrm{segno}(f(b_k))$ e
$s_k$ per $\mathrm{segno}(f (\frac {a_k +b_k}{2}))$, dove

\begin{equation}
    \mathrm{segno}(x) =
    \begin{cases}
        -1 & \text{se } x < 0, \\
         0 & \text{{se }} x = 0, \\
        +1 & \text{{se }} x > 0
    \end{cases}
\end{equation}

Per il calcolo di $\frac{a_k+b_k}{2}$ in virgola si deve usare la formula: $a_k + (b_k - a_k) / 2$ in modo da ridurre gli errori di troncamento.

Si vedano il listato \ref{lst:bisezione} e gli esempi.

\begin{lstfloat}
    \lstinputlisting{lst/bisezione.js}
    \caption{Descrizione in JavaScript del metodo di Bisezione}
    \label{lst:bisezione}
\end{lstfloat}

\begin{table}
    \begin{center}
\pgfplotstabletypeset[
	col sep=tab,
    every head row/.style={before row=\toprule,after row=\midrule},	% style the first row
	every last row/.style={after row=\bottomrule},	% style the last row
    every column/.style={dec sep align,precision=10}
	%every first column/.style={column type/.add={|}{}},	% style the first column
	%every last column/.style={column type/.add={}{|}},	% style the last column
	%columns/C/.style = {column type/.add={|}{|}}	% style the designated column
]{tbl/tab_bis_sqrt_6.dat}
\end{center}        
\caption[]{Metodo dicotomico applicato a $x^2 -6$ nell'intervallo $[0, 6]$ con nmax = 10}
\end{table}

% \pgfplotstableread{tbl/tab_bis_sqrt_6.dat}\tab_bis_sqrt_6
\begin{tikzpicture}
    \begin{axis}[
        axis lines = left,
        xlabel = \(x\),
        ylabel = {\(f(x)\)},
        grid=major, % Display a grid
        grid style={dashed,gray!30}, % Set the
        ]
        \addplot [
            domain=1:4,
            samples=100, 
            color=blue,
            ]
            {x^2 - 6};
        \addlegendentry{\(x^2 -6\)}
        \addplot[only marks] table [x=$x_k$, y=$f_k$, col sep=tab] {tbl/tab_bis_sqrt_6.dat};
    \end{axis}
\end{tikzpicture}

\begin{table}
    \begin{center}
        \pgfplotstabletypeset[
            col sep=tab,
            every head row/.style={before row=\toprule,after row=\midrule},	% style the first row
            every last row/.style={after row=\bottomrule},	% style the last row
            every column/.style={dec sep align,precision=10}
            %every first column/.style={column type/.add={|}{}},	% style the first column
            %every last column/.style={column type/.add={}{|}},	% style the last column
            %columns/C/.style = {column type/.add={|}{|}}	% style the designated column
        ]{tbl/tab_bis_sin.dat}
    \end{center}        
    \caption[]{Metodo dicotomico applicato a $sin(x)$ nell'intervallo $[3, 3.2]$ con nmax = 10}
\end{table}

\begin{table}
    \begin{center}
        \pgfplotstabletypeset[
            col sep=tab,
            every head row/.style={before row=\toprule,after row=\midrule},	% style the first row
            every last row/.style={after row=\bottomrule},	% style the last row
            every column/.style={dec sep align,precision=10}
            %columns/.style={sci,sci subscript,sci zerofill,dec sep align}
            %every first column/.style={column type/.add={|}{}},	% style the first column
            %every last column/.style={column type/.add={}{|}},	% style the last column
            %columns/C/.style = {column type/.add={|}{|}}	% style the designated column
        ]{tbl/tab_bis_exp_mx.dat}
    \end{center}        
    \caption[]{Metodo dicotomico applicato a $e^{e^{-x}}-x$ nell'intervallo $[0, 1]$ con nmax = 10}
\end{table}

\section{Iterazioni di punto fisso}

In generale si può costruire un metodo iterativo cercando un punto fisso di una funzione $\Phi(x)$,
costruita in moda che si annulli ne punto desiderato, un valore $\bar{x}$ tale che $\Phi(\bar{x}) = \bar{x}$.

Il punto fisso è calcolato tramite l'applicazione ripetuta della regola di ricorrenza:

$$x_{k+1} = \Phi(x_k)$$.

\subsection{Approssimazioni con rette}

Una retta è definita da una funzione del tipo $f(x) = m x + q$.
Se imponiamo il passaggio per il punto $(x_k, f(x_k))$ della funzione di cui cerchiamo una radice, il fascio di rette sarà:

$$f(x) - f(x_k) = m (x - x_k)$$.

Possiamo generare delle iterazioni andando a fissare il coefficiente angolare ad ogni iterazione e determinando l'intersezione della retta con l'asse delle ascisse.

$$f(x_{k+1}) - f(x_k) = m_k (x_{k+1} - x_k)$$

Risolvendo per $$f(x_{k+1}) = 0$$ si ha

\begin{equation}
    x_{k+1} = x_k - \frac{f(x_k)}{m_k}
\end{equation}

\begin{figure}
    \begin{center}
        \includestandalone{img/approx_rette}
        \caption{Approssimazione di una funzione con una retta passante per ($x_k$, $f(x_k)$)}
        \label{fig:retta_approx}
    \end{center}
\end{figure}


\subsubsection{Metodo delle corde}

Si considerino due punti $a=x_0$ e $b=x_1$ tali da soddisfare le ipotesi del teorema di Bolzano.
È possibile costruire una successione che per ogni $k \geq 0$  il punto $x_{k+1}$ sia lo zero della retta passante per il punto
($x_{k}$, $f(x_{k})$) e di coefficiente angolare

$$\displaystyle m_k = \frac{f(a)  - f(f_k)}{a - x_k}.$$

\subsubsection{Metodo delle secanti}

Dati due punti iniziali $x_0$ e $x_1$, si considera la secante passante per i due punti dati.

In generale il coefficiente angolare $m_k$ è calcolato come:

$$m_k = \frac{f(x_k) - f(x_{k-1})}{x_k - x_{k-1}}.$$

L'iterata è

$$x_{k+1} = x_k - f(x_k) \cdot \frac{x_k - x_{k-1}}{f(x_k) - f(x_{k-1})}.$$

\subsubsection{Metodo delle tangenti}

Si approssima la funzione $f(x)$ con la retta $r(x) - f(x_k) = f'(x_k) (x - x_k)$ tangente ad essa in $(x_k, f(x_k))$.

L'iterata assume la forma:

$$x_{k+1} = x_k - \frac{f(x_k)}{f'{x_k}}.$$

Si vedano il listato \ref{lst:tangenti} e gli esempi.

\begin{lstfloat}
    \lstinputlisting{lst/tangenti.js}
    \caption{Descrizione in JavaScript del metodo delle tangenti}
    \label{lst:tangenti}
\end{lstfloat}

\begin{table}
    \begin{center}
\pgfplotstabletypeset[
	col sep=tab,
    every head row/.style={before row=\toprule,after row=\midrule},	% style the first row
	every last row/.style={after row=\bottomrule},	% style the last row
    every column/.style={dec sep align,precision=10}
	%every first column/.style={column type/.add={|}{}},	% style the first column
	%every last column/.style={column type/.add={}{|}},	% style the last column
	%columns/C/.style = {column type/.add={|}{|}}	% style the designated column
]{tbl/tab_tan_sqrt_6.dat}
\end{center}        
\caption[]{Metodo delle tangenti applicato a $x^2 -6$ con stima iniziale 3 e nmax = 10}
\end{table}

% \pgfplotstableread{tbl/tab_bis_sqrt_6.dat}\tab_bis_sqrt_6
\begin{tikzpicture}
    \begin{axis}[
        axis lines = left,
        xlabel = \(x\),
        ylabel = {\(f(x)\)},
        grid=major, % Display a grid
        grid style={dashed,gray!30}, % Set the
        ]
        \addplot [
            domain=1:4,
            samples=100, 
            color=blue,
            ]
            {x^2 - 6};
        \addlegendentry{\(x^2 -6\)}
        \addplot[only marks] table [x=$x_k$, y=$f_k$, col sep=tab] {tbl/tab_tan_sqrt_6.dat};
    \end{axis}
\end{tikzpicture}

\begin{table}
    \begin{center}
        \pgfplotstabletypeset[
            col sep=tab,
            every head row/.style={before row=\toprule,after row=\midrule},	% style the first row
            every last row/.style={after row=\bottomrule},	% style the last row
            every column/.style={dec sep align,precision=10}
            %every first column/.style={column type/.add={|}{}},	% style the first column
            %every last column/.style={column type/.add={}{|}},	% style the last column
            %columns/C/.style = {column type/.add={|}{|}}	% style the designated column
        ]{tbl/tab_tan_sin.dat}
    \end{center}        
    \caption[]{Metodo delle tangenti applicato a $sin(x)$ con stima iniziale 3,1 e nmax = 10}
\end{table}

\begin{table}
    \begin{center}
        \pgfplotstabletypeset[
            col sep=tab,
            every head row/.style={before row=\toprule,after row=\midrule},	% style the first row
            every last row/.style={after row=\bottomrule},	% style the last row
            every column/.style={dec sep align,precision=10}
            %columns/.style={sci,sci subscript,sci zerofill,dec sep align}
            %every first column/.style={column type/.add={|}{}},	% style the first column
            %every last column/.style={column type/.add={}{|}},	% style the last column
            %columns/C/.style = {column type/.add={|}{|}}	% style the designated column
        ]{tbl/tab_tan_exp_mx.dat}
    \end{center}        
    \caption[]{Metodo delle tangenti applicato a $e^{e^{-x}}-x$ con stima iniziale 0,5 e nmax = 10}
\end{table}

\subsubsection{Esempio: algoritmo del reciproco}

Si vuole cercare il reciproco del valore $\nu$ come $\alpha = \frac{1}{\nu}$ con il metodo delle tangenti.

Per prima cosa occorre trasformare il problema con una funzione che si annulla in $\frac{1}{\nu}$.

Scegliamo $$f(x) = \nu - \frac{1}{x}$$ che applicata al reciproco di $\nu$ produce
$f(\frac{1}{\nu}) = \nu - \frac{1}{\frac{1}{\nu}} = \nu - \nu = 0$. 

La derivata prima assume la forma $f'(x) = \frac{1}{x^2}$ e

$$\frac{f(x)}{f'(x)} = \frac{\nu - \frac{1}{x}}{\frac{1}{x^2}} = \nu x^2 - x.$$

L'iterata del metodo delle tangenti è:

$$x_{k+1} = x_k - (\nu x_k^2 - x_k) = 2 x_k - \nu x_k^2  = x_k \cdot (2 - \nu \cdot x_k).$$

Si noti che per calcolare il reciproco di un numero sono sufficienti le operazioni di moltiplicazione e sottrazione.
Per la moltiplicazione occorrono le operazioni primitive di scorrimento e addizione e per la sottrazione quelle di negazione bit a bit e di addizione (basterebbe il solo incremento unitario).

Nota: queste proprietà permettono di realizzare le CPU senza l'operazione di divisione.

\subsection{Approssimazioni con parabole}

\subsubsection{Metodo di Newton del secondo ordine}



\begin{equation}
    p (x) = a x^2 + b x + c
\end{equation}


\end{document}