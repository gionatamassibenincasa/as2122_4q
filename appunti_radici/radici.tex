 \documentclass[10pt]{article} 
\usepackage[utf8]{inputenc}
\usepackage{geometry}
\geometry{a4paper}
\geometry{top=0.5in, bottom=0.5in, left=.7in, right=.7in}
%\usepackage{blindtext}
\usepackage{hyperref}
\usepackage{bookmark}
\usepackage{booktabs} % for much better looking tables
\usepackage{array} % for better arrays (eg matrices) in maths
\usepackage{paralist} % very flexible & customisable lists (eg. enumerate/itemize, etc.)
\usepackage{verbatim} % adds environment for commenting out blocks of text & for better verbatim
\usepackage{subfig} % make it possible to include more than one captioned figure/table in a single float
% These packages are all incorporated in the memoir class to one degree or another...
\usepackage[italian]{babel}

%%% HEADERS & FOOTERS
\usepackage{fancyhdr} % This should be set AFTER setting up the page geometry
\pagestyle{fancy} % options: empty , plain , fancy
\renewcommand{\headrulewidth}{0pt} % customise the layout...
\lhead{}\chead{}\rhead{}
\lfoot{}\cfoot{\thepage}\rfoot{}

%%% SECTION TITLE APPEARANCE
\usepackage{sectsty}
\allsectionsfont{\sffamily\mdseries\upshape} % (See the fntguide.pdf for font help)
% (This matches ConTeXt defaults)

%%% ToC (table of contents) APPEARANCE
\usepackage[nottoc,notlof,notlot]{tocbibind} % Put the bibliography in the ToC
\usepackage[titles,subfigure]{tocloft} % Alter the style of the Table of Contents
\renewcommand{\cftsecfont}{\rmfamily\mdseries\upshape}
\renewcommand{\cftsecpagefont}{\rmfamily\mdseries\upshape} % No bold!

\newtheorem{theorem}{Teorema}
\usepackage{amsmath}
\usepackage{amsfonts}
\usepackage{pgfplots}
\pgfplotsset{width=10cm,compat=1.9}

\usepackage{multirow}
% \usepackage[keeptemps]{pythontex}
\usepackage{listings}
\lstdefinelanguage{JavaScript}{
  keywords={typeof, new, true, false, catch, function, return, null, catch, switch, var, if, in, while, do, else, case, break},
  keywordstyle=\color{blue}\bfseries,
  ndkeywords={class, export, boolean, throw, implements, import, this},
  ndkeywordstyle=\color{darkgray}\bfseries,
  identifierstyle=\color{black},
  sensitive=false,
  comment=[l]{//},
  morecomment=[s]{/*}{*/},
  commentstyle=\color{purple}\ttfamily,
  stringstyle=\color{red}\ttfamily,
  morestring=[b]',
  morestring=[b]"
}

\lstset{
   language=JavaScript,
   backgroundcolor=\color{lightgray},
   extendedchars=true,
   basicstyle=\footnotesize\ttfamily,
   showstringspaces=false,
   showspaces=false,
   numbers=left,
   numberstyle=\footnotesize,
   numbersep=9pt,
   tabsize=2,
   breaklines=true,
   showtabs=false,
   captionpos=b
}


%%% The "real" document content comes below...

\title{Alcuni metodi iterativi per la ricerca di radici di funzioni}
\author{Gionata Massi}
\date{} % Activate to display a given date or no date (if empty),
         % otherwise the current date is printed 

\begin{document}
\maketitle

\thispagestyle{empty}%\frenchspacing

\section{Il problema}

Data una funzione $f : \mathbb{R} \to \mathbb{R}$, determinare un valore reale $\alpha$ tale che $f(\alpha) = 0$.

Usualmente consideriamo funzioni continue in $\mathbb{R}$ o almeno in un intevallo $[a, b] \subseteq \mathbb{R}$ chiuso e limitato in cui ricercare una radice.

\section{Esistenza delle radici}

Non tutte le funzioni ammettono radici, ad esempio $x \mapsto k$ e $x \mapsto (x + k)^2$, dove $k \neq 0$ (es: fig. \ref{fig:nozeri}).

\begin{figure}[ht]
\centering
\begin{tikzpicture}
\begin{axis}[
    axis lines = left,
    xlabel = \(x\),
    ylabel = {\(f(x)\)},
    grid=both,
    ymin = -10
]
%Below the red parabola is defined
\addplot [
    domain=-3:3, 
    samples=100, 
    color=red,
]
{x^2 + 1};
\addlegendentry{\(x^2 + 1\)}
%Here the blue constant function
\addplot [
    domain=-3:3, 
    samples=100, 
    color=blue,
    ]
    {-4};
\addlegendentry{\(-4\)}

\end{axis}
\end{tikzpicture}
\caption{Funzioni che non intersecano l'asse $y = 0$}
\label{fig:nozeri}
\end{figure}

Altre funzioni hanno radici nei punti di massimo o di minimo locale (es: fig. \ref{fig:zero_estremante}).

\begin{figure}[ht]
    \centering
    \begin{tikzpicture}
    \begin{axis}[
        axis lines = left,
        xlabel = \(x\),
        ylabel = {\(f(x)\)},
        grid=both
    ]
    \addplot [
        domain=-10:10, 
        samples=100, 
        color=red,
    ]
    {cos(deg(x))-1};
    \addlegendentry{\(cos(x) - 1\)}
    \addplot [
        domain=-10:10, 
        samples=100, 
        color=blue,
        ]
        {(x/5)^2};
    \addlegendentry{\(\left(\frac{x}{5}\right)^2\)}
    \end{axis}
    \end{tikzpicture}
    \caption{Funzioni che intersecano l'asse $y = 0$ in un estremante}
    \label{fig:zero_estremante}
    \end{figure}
    

Per essere sicuri che una funzione ammetta almeno una radice richiediamo che la funzione assuma valori positivi e negativi in un certo intervallo e che sia continua.

\begin{theorem}[Bolzano]
Se $f (x)$ è una funzione continua sull'intervallo limitato e chiuso $[a, b]$ e $f (a) \cdot f (b) < 0$, allora esiste almeno una radice di $f (x)$ nell'intervallo $[a, b]$.
\end{theorem}

Se le ipotesi del teorema sono vere può esistere una sola radice oppure ce ne possono essere in numero finito o anche infinite (fig. \ref{fig:ipotesibolzano}).

\begin{figure}[ht]
    \centering
    \begin{tikzpicture}
    \begin{axis}[
        axis lines = left,
        xlabel = \(x\),
        ylabel = {\(f(x)\)},
        grid=both
    ]
    \addplot [
        domain=-1:1, 
        samples=100, 
        color=red,
    ]
    {exp(x)-3*x};
    \addlegendentry{\(e^x-3x\)}
    \addplot [
        domain=-1:1, 
        samples=100, 
        color=blue,
        ]
        {(x-1/3)^3};
    \addlegendentry{\(\left(x-\frac{1}{3}\right)^3\)}
    \addplot [
        domain=-1:1, 
        samples=100, 
        color=green,
        ]
        {sin(deg(10*x^2))/(5*x)};
    \addlegendentry{\(\frac{\sin(10 x^2)}{5x}\)}
    \addplot [
        domain=-1:1, 
        samples=400, 
        color=purple,
        ]
        {sin(deg(1/x))};
    \addlegendentry{\(\sin(\frac{1}{x})\)}
    \end{axis}
    \end{tikzpicture}
    \caption{Funzioni che assumono valori opposti in $[-1, 1]$}
    \label{fig:ipotesibolzano}
    \end{figure}


Un metodo di ricerca delle radici, se convergente, restituirà una sola delle radici.
Si intuisce che maggiore è la pendenza della funzione in un intorno della radice, più è facile discriminare la radice. Se invece la pendenza è nulla o quasi, allora il problema si dice mal condizionato.

\section{Metodi diretti e metodi iterativi}


I \textbf{metodi diretti} sono algoritmi che, in assenza di errori di arrotondamento, forniscono la soluzione in un numero finito di operazioni.

I \textbf{metodi iterativi} sono algoritmi nei quali la soluzione è ottenuta come limite di una successione di soluzioni. Nella risposta fornita da un metodo iterativo è, quindi, presente usualmente un errore di troncamento.

Si veda Comincioli2004.

\section{Metodo di bisezione}

Sia $f (x)$ una funzione continua sull'intervallo limitato e chiuso $[a, b]$ con $f (a) \cdot f (b) < 0$. L'algoritmo genera una successione di intervalli
$[a_k , b_k]$ con $f (a_k ) \cdot f (b_k ) < 0$ e con $[a_k , b_k ] \subset [a_{k-1}$ , $b_{k-1} ]$ e $|b_k - a_k | = \frac{1}{2}|b_{k-1} - a_{k-1} |$. 

Date due tolleranze $\epsilon_1$ , $\epsilon_2$ , l’algoritmo si arresta o quando $|b_k - a_k | \leq \epsilon_1$ o quando $|f (\frac {a_k +b_k}{2} )| \leq
\epsilon_2$ o infine quando $k > \text{nmax}$, ove nmax è un numero massimo di iterazioni fissato.

Per alleggerire la notazione usiamo $s_{a_k}$ per indicare $\mathrm{segno} \left(f\left(a_k\right)\right)$,
$s_{b_k}$ per $\mathrm{segno}(f(b_k))$ e
$s_k$ per $\mathrm{segno}(f (\frac {a_k +b_k}{2}))$, dove

\begin{equation}
    \mathrm{segno}(x) =
    \begin{cases}
        -1 & \text{se } x < 0, \\
         0 & \text{{se }} x = 0, \\
        +1 & \text{{se }} x > 0
    \end{cases}
\end{equation}

Per il calcolo di $\frac{a_k+b_k}{2}$ in virgola si deve usare la formula: $a_k + (b_k - a_k) / 2$ in modo da ridurre gli errori di troncamento.

\begin{table}
    \begin{center}
        \begin{tabular}{|c|*{7}{l|}}\hline
        \multicolumn{8}{|c|}{\bf Iterazioni del metodo di bisezione}\\\hline
        \multicolumn{1}{|c|}{$k$}&
        \multicolumn{1}{|c|}{$a_k$}&
        \multicolumn{1}{|c|}{$b_k$}&
        \multicolumn{1}{|c|}{$x_k$}&
        \multicolumn{1}{|c|}{$s_{a_k}$}&
        \multicolumn{1}{|c|}{$s_{b_k}$}&
        \multicolumn{1}{|c|}{$s_k$}&
        \multicolumn{1}{|c|}{$|b_{k} - a_{k}|$}\\\hline
        0&0&6&3&-1&1&1&6\\
        1&0&3&1.5&-1&1&-1&3\\
        2&1.5&3&2.25&-1&1&-1&1.5\\
        3&2.25&3&2.625&-1&1&1&0.75\\
        4&2.25&2.625&2.4375&-1&1&-1&0.375\\
        5&2.4375&2.625&2.53125&-1&1&1&0.1875\\
        6&2.4375&2.53125&2.484375&-1&1&1&0.09375\\
        7&2.4375&2.484375&2.4609375&-1&1&1&0.046875\\
        8&2.4375&2.4609375&2.44921875&-1&1&-1&0.0234375\\
        9&2.44921875&2.4609375&2.455078125&-1&1&1&0.01171875\\
        \hline
        \end{tabular}
    \end{center}        
    \caption[]{Metodo dicotomico applicato a $x^2 -6$ nell'intervallo $[0, 6]$ con nmax = 10}
\end{table}

\begin{lstlisting}[float]
/**
 *
 * @param {Function} f una funzione continua in [a, b]
 * @param {Number} a l'estremo sinistro dell'intervallo di incertezza
 * @param {Number} b l'estremo destro dell'intervallo di incertezza
 * @param {Number} e1 tolleranza su asse delle ascisse
 * @param {Number} e2 tolleranza su asse delle ordinate
 * @param {Number} nmax numero massimo di iterazioni
 */
const bisezione = (f, a, b, e1 = 1e-16, e2 = 1e-16, nmax = 10) => {
  let f_a = f(a);
  let f_b = f(b);
  let s_a = Math.sign(f_a);
  let s_b = Math.sign(f_b);
  if (s_a === s_b) {
    throw "Segni concordi nei due estremi.";
  }
  let x;
  for (let iter = 0; iter < nmax && b - a >= e1; iter++) {
    x = a + (b - a) / 2;
    let f_x = f(x);
    if (Math.abs(f_x) < e2) {
      return x;
    }
    let s_x = Math.sign(f_x);
    if (s_a === s_x) {
      a = x;
    } else {
      b = x;
    }
  }
  return x;
};
\end{lstlisting}

\section{Iterazioni di punto fisso}

In generale si può costruire un metodo iterativo cercando un punto fisso di $\Phi(x)$, cioè un valore $\bar{x}$ tale che $\Phi(\bar{x}) = \bar{x}$.

$$x_{k+1} = \Phi(x_k)$.

\subsection{Approssimazioni con rette}

Una retta è definita da una funzione del tipo $f(x) = m x + q$.
Se imponiamo il passaggio per il punto $(x_k, f(x_k))$ della funzione di cui cerchiamo una radice, il fascio di rette sarà:

$$f(x) - f(x_k) = m (x - x_k)$$.

Possiamo generare delle iterazioni andando a fissare il coefficiente angolare ad ogni iterazione e determinando l'intersezione della retta con l'asse delle ascisse.


$$f(x_{k+1}) - f(x_k) = m_k (x_{k+1} - x_k)$$

Risolvendo per $$f(x_{k+1}) = 0$$ si ha

\begin{equation}
    x_{k+1} = x_k - \frac{f(x_k)}{m_k}
\end{equation}

\subsubsection{Metodo delle corde}

\subsubsection{Metodo delle secanti}

\subsubsection{Metodo delle tangenti}

\subsection{Approssimazioni con parabole}

\subsubsection{Metodo di Newton del secondo ordine}



\begin{equation}
    p (x) = a x^2 + b x + c
\end{equation}


\end{document}