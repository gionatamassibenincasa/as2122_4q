\subsection{Metodo delle secanti}

Dati due punti iniziali $x_0$ e $x_1$, si considera la secante passante per i due punti dati.

In generale il coefficiente angolare $m_k$ è calcolato come:

$$m_k = \frac{f(x_k) - f(x_{k-1})}{x_k - x_{k-1}}.$$

L'iterata è

$$x_{k+1} = x_k - f(x_k) \cdot \frac{x_k - x_{k-1}}{f(x_k) - f(x_{k-1})}.$$

\subsubsection{Convergenza}

Il metodo non necessariamente converge.
Può oscillare o divergere.

\subsubsection{Codifica in JavaScript}

\begin{lstfloat}
    \lstinputlisting{lst/secanti.js}
    \caption{Descrizione in JavaScript del metodo delle secanti}
    \label{lst:secanti}
\end{lstfloat}

\subsubsection{Esempi}

Si vedano gli esempi nelle tabb.~\ref{tbl:tab_sec_sqrt_6}, \ref{tbl:tab_sec_sin} e \ref{tbl:tab_sec_exp_mx} e la fig.~\ref{fig:sec_sqrt_6}.


\begin{table}
    \begin{center}
\pgfplotstabletypeset[
	col sep=tab,
    every head row/.style={before row=\toprule,after row=\midrule},	% style the first row
	every last row/.style={after row=\bottomrule},	% style the last row
    every column/.style={dec sep align,precision=10}
]{tbl/tab_bis_sqrt_6.dat}
\end{center}        
\caption[]{Metodo delle secanti applicato a $x^2 -6$ nell'intervallo $[0, 6]$ con nmax = 10}
\label{tbl:tab_sec_sqrt_6}
\end{table}

\begin{figure}[ht]
    \centering
    \includestandalone{img/iter_cor_sqrt_6}
    \caption{Successione delle soluzioni del metodo delle secanti applicato a $x^2 -6$ nell'intervallo $[0, 6]$}
    \label{fig:sec_sqrt_6}
\end{figure}

\begin{table}
    \begin{center}
        \pgfplotstabletypeset[
            col sep=tab,
            every head row/.style={before row=\toprule,after row=\midrule},	% style the first row
            every last row/.style={after row=\bottomrule},	% style the last row
            every column/.style={dec sep align,precision=10}
        ]{tbl/tab_cor_sin.dat}
    \end{center}        
    \caption[]{Metodo delle secanti applicato a $sin(x)$ nell'intervallo $[3, 3.2]$ con nmax = 10}
    \label{tbl:tab_sec_sin}
\end{table}

\begin{table}
    \begin{center}
        \pgfplotstabletypeset[
            col sep=tab,
            every head row/.style={before row=\toprule,after row=\midrule},	% style the first row
            every last row/.style={after row=\bottomrule},	% style the last row
            every column/.style={dec sep align,precision=10}
            %columns/.style={sci,sci subscript,sci zerofill,dec sep align}
            %every first column/.style={column type/.add={|}{}},	% style the first column
            %every last column/.style={column type/.add={}{|}},	% style the last column
            %columns/C/.style = {column type/.add={|}{|}}	% style the designated column
        ]{tbl/tab_cor_exp_mx.dat}
    \end{center}        
    \caption[]{Metodo delle secanti applicato a $e^{e^{-x}}-x$ nell'intervallo $[0, 1]$ con nmax = 10}
    \label{tbl:tab_sec_exp_mx}
\end{table}
